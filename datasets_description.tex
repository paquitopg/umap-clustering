We evaluate our UMAP implementation on three diverse datasets that span different domains and scales. The \textbf{Forest Covertype} dataset contains approximately 581,000 samples with 54 features, representing different forest cover types based on cartographic variables. This dataset is primarily used to assess the clustering performance of dimensionality reduction methods, as it contains well-defined classes that should form distinct clusters in the embedding space. The \textbf{Fashion-MNIST} dataset consists of 70,000 grayscale images of fashion items across 10 classes, with each image represented as a 784-dimensional vector. This dataset serves multiple purposes: we use it for qualitative visual comparisons between different dimensionality reduction techniques, to evaluate intrinsic quality measures such as trustworthiness and continuity, and to assess downstream clustering performance across different embedding dimensions. Finally, the \textbf{Mini-BooNE} dataset is a physics dataset from particle physics experiments, containing approximately 130,000 instances with 50 features, used to distinguish between electron neutrinos (signal) and muon neutrinos (background). This dataset is particularly valuable for evaluating the computational scalability of our algorithms, as it allows us to test performance across a wide range of sample sizes and embedding dimensions, demonstrating how different methods scale with both the number of data points and the target dimensionality.
